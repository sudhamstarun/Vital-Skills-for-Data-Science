% Options for packages loaded elsewhere
\PassOptionsToPackage{unicode}{hyperref}
\PassOptionsToPackage{hyphens}{url}
%
\documentclass[
]{article}
\usepackage{amsmath,amssymb}
\usepackage{iftex}
\ifPDFTeX
  \usepackage[T1]{fontenc}
  \usepackage[utf8]{inputenc}
  \usepackage{textcomp} % provide euro and other symbols
\else % if luatex or xetex
  \usepackage{unicode-math} % this also loads fontspec
  \defaultfontfeatures{Scale=MatchLowercase}
  \defaultfontfeatures[\rmfamily]{Ligatures=TeX,Scale=1}
\fi
\usepackage{lmodern}
\ifPDFTeX\else
  % xetex/luatex font selection
\fi
% Use upquote if available, for straight quotes in verbatim environments
\IfFileExists{upquote.sty}{\usepackage{upquote}}{}
\IfFileExists{microtype.sty}{% use microtype if available
  \usepackage[]{microtype}
  \UseMicrotypeSet[protrusion]{basicmath} % disable protrusion for tt fonts
}{}
\makeatletter
\@ifundefined{KOMAClassName}{% if non-KOMA class
  \IfFileExists{parskip.sty}{%
    \usepackage{parskip}
  }{% else
    \setlength{\parindent}{0pt}
    \setlength{\parskip}{6pt plus 2pt minus 1pt}}
}{% if KOMA class
  \KOMAoptions{parskip=half}}
\makeatother
\usepackage{xcolor}
\usepackage[margin=1in]{geometry}
\usepackage{graphicx}
\makeatletter
\def\maxwidth{\ifdim\Gin@nat@width>\linewidth\linewidth\else\Gin@nat@width\fi}
\def\maxheight{\ifdim\Gin@nat@height>\textheight\textheight\else\Gin@nat@height\fi}
\makeatother
% Scale images if necessary, so that they will not overflow the page
% margins by default, and it is still possible to overwrite the defaults
% using explicit options in \includegraphics[width, height, ...]{}
\setkeys{Gin}{width=\maxwidth,height=\maxheight,keepaspectratio}
% Set default figure placement to htbp
\makeatletter
\def\fps@figure{htbp}
\makeatother
\setlength{\emergencystretch}{3em} % prevent overfull lines
\providecommand{\tightlist}{%
  \setlength{\itemsep}{0pt}\setlength{\parskip}{0pt}}
\setcounter{secnumdepth}{-\maxdimen} % remove section numbering
\ifLuaTeX
  \usepackage{selnolig}  % disable illegal ligatures
\fi
\usepackage{bookmark}
\IfFileExists{xurl.sty}{\usepackage{xurl}}{} % add URL line breaks if available
\urlstyle{same}
\hypersetup{
  pdftitle={NYPD Shooting Data Report based on Age, Gender, Race, and other factors},
  pdfauthor={Tarun Sudhams},
  hidelinks,
  pdfcreator={LaTeX via pandoc}}

\title{NYPD Shooting Data Report based on Age, Gender, Race, and other
factors}
\author{Tarun Sudhams}
\date{2024-11-23}

\begin{document}
\maketitle

\section{Introduction}\label{introduction}

The NYPD dataset has multiple factors that can be analyzed and
visualized to get a better understanding of the data that we are looking
at. This includes factors like victim's age, sex and race. This
visualization aims at understanding the patterns and trends of gun
violence in New York City.

\section{Import required libraries}\label{import-required-libraries}

These are the two main libraries that we would be using for
visualization and data wrangling.

library(ggplot2) library(tidyverse) library(dplyr)

\section{Import the dataset}\label{import-the-dataset}

dataset\_url \textless-
``\url{https://data.cityofnewyork.us/api/views/833y-fsy8/rows.csv}''
dataset \textless- read\_csv(dataset\_url)

\section{Explore the dataset}\label{explore-the-dataset}

Before we start tidying and transforming the dataset, let's take a look
at how the dataset looks like and what each column looks like and the
sample data points in each of the columns.

glimpse(dataset)

This gives up a glimpse at the kind of datatypes each column contains.
Now with this information, we can try and look at following factors in
the subsequent sections:

\begin{itemize}
\tightlist
\item
  Geographical Distribution
\item
  Monthly Shooting Trends
\item
  Victims Demographics by Sex
\item
  Weapon Types
\item
  Socioeconomic Analysis
\item
  Location Types
\end{itemize}

But before we move on to that, we should focus on data cleaning,
processing to get it ready for analysis and visualizations.

\section{Data Preprocessing}\label{data-preprocessing}

\begin{enumerate}
\def\labelenumi{\arabic{enumi}.}
\tightlist
\item
  Handle missing values in our critical variables {[}age, sex, race,
  outcome{]}:
\end{enumerate}

dataset \textless- dataset
\%\textgreater\%filter(!is.na(VIC\_AGE\_GROUP), !is.na(VIC\_SEX),
!is.na(VIC\_RACE))

\begin{enumerate}
\def\labelenumi{\arabic{enumi}.}
\setcounter{enumi}{1}
\tightlist
\item
  Convert the datatypes of columns
\end{enumerate}

dataset\(VIC_AGE_GROUP <- as.factor(dataset\)VIC\_AGE\_GROUP)
dataset\(VIC_SEX <- as.factor(dataset\)VIC\_SEX)
dataset\(VIC_RACE <- as.factor(dataset\)VIC\_RACE)
dataset\(OCCUR_DATE <- as.Date(dataset\)OCCUR\_DATE, ``\%m/\%d/\%Y'')

\begin{enumerate}
\def\labelenumi{\arabic{enumi}.}
\setcounter{enumi}{2}
\tightlist
\item
  Create a new variable for time of day based on incident time
\end{enumerate}

dataset \textless- dataset \%\textgreater\%
filter(!is.na(VIC\_AGE\_GROUP), !is.na(VIC\_SEX), !is.na(VIC\_RACE))
\%\textgreater\% mutate( VIC\_AGE\_GROUP = as.factor(VIC\_AGE\_GROUP),
VIC\_SEX = as.factor(VIC\_SEX), VIC\_RACE = as.factor(VIC\_RACE),
OCCUR\_DATE = as.Date(OCCUR\_DATE), MONTH = format(OCCUR\_DATE, `\%m'),
time\_of\_day = cut(as.POSIXlt(OCCUR\_TIME, format=`\%H:\%M:\%S')\$hour,
breaks=c(-1, 6, 12, 18, 24), labels=c(`Night', `Morning', `Afternoon',
`Evening')) )

\section{Data Visualization}\label{data-visualization}

\subsection{Geographic Distribution
(Heatmap)}\label{geographic-distribution-heatmap}

We already have some location data that we could use to generate a
heatmap to see which areas are more prone to gun violence

\subsubsection{Create a heatmap}\label{create-a-heatmap}

ggplot(dataset, aes(x = Longitude, y = Latitude)) +
stat\_density\_2d(aes(fill = ..level..), geom = ``polygon'', alpha =
0.5) + scale\_fill\_gradient(low = ``lightblue'', high = ``darkblue'') +
labs(title = ``Geographical Distribution of Shooting Incidents'', x =
``Longitude'', y = ``Latitude'') + theme\_minimal()

Now that we have a heatmap, let's find the top 5 locations based on this
heatmap data.

\subsubsection{Count the number of incidents per
location}\label{count-the-number-of-incidents-per-location}

top\_locations \textless- dataset \%\textgreater\% group\_by(BORO)
\%\textgreater\% summarise(count = n()) \%\textgreater\%
arrange(desc(count))

\subsubsection{Display the top
locations}\label{display-the-top-locations}

print(top\_locations)

This makes it clear that Brooklyn is probably the most dangerous area in
all of New York City in terms of gun violence followed by Bronx, Queens,
Manhattan and Staten Island.

\subsection{Monthly Shooting Trends (Line
Chart)}\label{monthly-shooting-trends-line-chart}

Now, let's take a look at the monthly shooting numbers by aggregating it
on a monthly basis:

\subsubsection{Aggregate data by month}\label{aggregate-data-by-month}

monthly\_shootings \textless- dataset \%\textgreater\% group\_by(MONTH)
\%\textgreater\% summarise(count = n())

\subsubsection{Create a line chart for monthly
shootings}\label{create-a-line-chart-for-monthly-shootings}

ggplot(monthly\_shootings, aes(x = MONTH, y = count)) +
geom\_line(color=``blue'') + geom\_point(color=``blue'') +
labs(title=``Monthly Shooting Incidents Over Time'', x=``Month'',
y=``Number of Shootings'') + theme\_minimal() +
theme(axis.text.x=element\_text(angle=45))

So the chart actually shows that there is steady rise in shootings in
the middle of the year and it shootings reduce in number towards the
later part of the year. That's an interesting observation although this
might not actually mean anything.

\subsection{Victim/Perpetrator Demographics by Sex (Bar
Chart)}\label{victimperpetrator-demographics-by-sex-bar-chart}

Let's begin with preparing the data for victim's sex and perperator's
sex to analyse which sex causes more violence and is subjected to gun
violence in the city.

\subsubsection{Prep data based on victom and perpetrator's
sex}\label{prep-data-based-on-victom-and-perpetrators-sex}

victim\_sex \textless- dataset \%\textgreater\% group\_by(VIC\_SEX)
\%\textgreater\% summarise(count=n())

perpetrator\_sex \textless- dataset \%\textgreater\%
group\_by(PERP\_SEX) \%\textgreater\% summarise(count=n())

\subsubsection{Create a bar chart}\label{create-a-bar-chart}

ggplot(victim\_sex, aes(x=VIC\_SEX, y=count, fill=VIC\_SEX)) +
geom\_bar(stat=``identity'') + labs(title=``Victim Demographics by
Sex'', x=``Sex'', y=``Number of Victims'') + theme\_minimal()

ggplot(perpetrator\_sex, aes(x=PERP\_SEX, y=count, fill=PERP\_SEX)) +
geom\_bar(stat=``identity'') + labs(title=``Perpetrator Demographics by
Sex'', x=``Sex'', y=``Number of Perpetrators'') + theme\_minimal()

It's quite clear that victims are mostly males howvever we can't draw
the same conclusion about Prepetrator's sex since there are a lot of
null and NA values also in the data we makes it difficult to come to a
conclusion.

\subsection{Victim/Perpetrator's Demography by
Race}\label{victimperpetrators-demography-by-race}

We also have some information on the kind of weapons used which can be
useful to deduce and understand what kinds of weapons caused the most
harm.

\subsubsection{Group dataset by victim's and perpetrator's
race}\label{group-dataset-by-victims-and-perpetrators-race}

victim\_race \textless- dataset \%\textgreater\% group\_by(VIC\_RACE)
\%\textgreater\% summarise(count=n())

perpetrators\_race \textless- dataset \%\textgreater\%
group\_by(PERP\_RACE) \%\textgreater\% summarise(count=n())

\subsubsection{Create a bar chart for victim's
race}\label{create-a-bar-chart-for-victims-race}

ggplot(victim\_race, aes(x=VIC\_RACE, y=count, fill=VIC\_RACE)) +
geom\_bar(stat=``identity'') + labs(title=``Victim Demographics by
Race'', x=``Sex'', y=``Number of Victims'') + theme\_minimal()

\subsubsection{Create a bar chart for perpetrator's
race}\label{create-a-bar-chart-for-perpetrators-race}

ggplot(perpetrators\_race, aes(x=PERP\_RACE, y=count, fill=PERP\_RACE))
+ geom\_bar(stat=``identity'') + labs(title=``Perpetrator's Demographics
by Race'', x=``Race'', y=``Number of Victims'') + theme\_minimal()

Similar conclusions can be drawn here about the race of the
victim/perpetrator. We can see that there is a lot of cases where the
data about the race is missing for perpetrator's which makes it
difficult to say which race is the most violent. More importantly,
socioeconomic factors also play an important role in determining whether
the race of the victim/perpetrator should actually matter. In our case,
it makes more sense to attribute it to socioeconomic factors rather than
simply making a conclusion based on the race of the perpetrator or the
victim.

\end{document}
